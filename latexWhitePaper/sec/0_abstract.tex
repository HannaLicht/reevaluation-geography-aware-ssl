\begin{abstract}
%The ABSTRACT is to be in fully justified italicized text, at the top of the left-hand column, below the author and affiliation information.
%Use the word ``Abstract'' as the title, in 12-point Times, boldface type, centered relative to the column, initially capitalized.
%The abstract is to be in 10-point, single-spaced type.
%Leave two blank lines after the Abstract, then begin the main text.
%Look at previous \confName~abstracts to get a feel for style and length.

Foundation models, trained on large datasets in a self-supervised manner, have become useful tools in artificial intelligence, especially in domains with abundant unlabeled data but scarce labeled data, such as remote sensing. This study re-evaluates the MoCo-v2 framework and compares it with a geography-aware contrastive learning approach introduced by Ayush et al. in the paper "Geography-Aware Self-Supervised Learning" \cite{geoAwareSelfSuper}. This approach leverages spatio-temporal structures in remote-sensing data and combines it with a geo-location classification pre-text task. I used the BigEarthNet dataset to evaulate the expressivness of the models' feature representations in several experiments. The experiments demonstrate that integrating metadata, particularly geo-location information, into foundation models is likely to improve performance in multi-label classification tasks. 

\end{abstract}